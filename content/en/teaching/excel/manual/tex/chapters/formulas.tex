% Autor: Alfredo Sánchez Alberca (email:asalber@ceu.es)

\chapter{Calculus with formulas}

Spreadsheets are used mainly for doing calculations and one of the most powerful features of spreadsheets are
calculation formulas. In this section we will see how to use them.

\section{Enter formulas}\hypertarget{enter-formulas}{}\label{enter-formulas}

To enter a formula in a cell always start typing an equal sign \texttt{=} and then the formula expression.

Formula expressions can contain arithmetic operators: addition \texttt{+}, subtraction \texttt{-}, multiplication
\texttt{*}, division \texttt{/} and powers \texttt{\^{}} and named predefined functions like \texttt{SUM}, \texttt{EXP},
\texttt{SIN}, etc. This allow to use Excel as a calculator. When Excel evaluates expressions first evaluate named
functions, then powers, then products and quotients, and finally additions and subtractions, but it's possible to use
parenthesis to force the evaluation of a subexpression before.

\textbf{Example} Assuming that cells A1, B1 and C1 contain the values 6, 3 and 2 respectively, the next table shows some
formulas and their respective results.

\begin{longtable}{cc}
\toprule
Formula & Result\\
\midrule
\endfirsthead
A1+B1-C1 & 7\\
A1+B1*C1 & 12\\
(A1+B1)*C1 & 18\\
A1/B1-C1 & 0\\
A1/(B1-C1) & 6\\
A1+B1\^{}C1 & 15\\
(A1+B1)\^{}C1 & 81\\
\bottomrule
\end{longtable}



\textbf{Example}. This \href{http://aprendeconalf.es/office/excel/manual/img/example_enter_formulas.gif}{animation} shows how to enter the formula 4+2 in cell A1, the formula 4-2 in cell B1, the formula 4*2 in cell C1, the formula 4/2 in cell D1, the formula 4\^{}2 in cell E1 and the formula ((4+1)*2)\^{}3 in cell F1.

\section{Using relative and absolutes cell references in formulas}\hypertarget{using-relative-and-absolutes-cell-references-in-formulas}{}\label{using-relative-and-absolutes-cell-references-in-formulas}

Formula expressions can content references to cells. When Excel evaluates formulas it replace every cell reference by its content before doing the calculation.

\textbf{Example}. This
\href{http://aprendeconalf.es/office/excel/manual/img/example_formulas_with_references.gif}{animation} shows how to use
the formula \texttt{=A1+B1} to add up the content of cells A1 and B1 in cell C1.

References that are formed by the name of the cell or range are known as \emph{relative references}, because referenced
cells change when you copy a cell with a formula and paste in another cell. In general, when you copy a formula $n$
columns to the right and $m$ rows down, the referenced cells in the formulas will be updated by the cells $n$ columns to
the right and $m$ rows down, an the same if you copy the cell to the left or top.

\textbf{Example}. This \href{http://aprendeconalf.es/office/excel/manual/img/example_copying_formulas_with_relative_references.gif}{animation} shows how to copy the formula \texttt{=A1+B1} in cell C1, with relative references to A1 and B1, to the cell E4, that is 2 columns to the right and 3 rows down. Observe how the formula in cell E4 is updated to \texttt{=C4+D4}.

A common way of copying the formula of a cell to adjacent cells is clicking the bottom-right corner of the cell and
dragging the cursor to the desired range of cells.

\textbf{Example}. This \href{http://aprendeconalf.es/office/excel/manual/img/example_fibonacci_serie.gif}{animation}
shows how to generate the first ten numbers of the Fibonacci sequence. Cells A1 and B1 contains the two first numbers of
the serie and cell C1 the formula \texttt{=A1+B1} that add the two first numbers up and gives the third number of the
serie. For generating the rest of the serie it is enough to copy the formula of cell C1 to the range D1:J1. Observe how
references in formulas of these cells are updated.

Although relative references are very helpful in many cases, sometimes we need the references in a formula to remain
fixed when copied elsewhere.
In that case we need to use \emph{absolute references}, that are like relative references but preceding the column name
or the row name with a \$ sign to fix either the row, the column or both on any cell reference.

\textbf{Example}. This
\href{http://aprendeconalf.es/office/excel/manual/img/example_copying_formulas_with_absolute_references.gif}{animation} shows how to calculate the IVA of a list of prices. Cells A2 to A5 contains the prices and cell F1 contains the IVA percentage. For calculating the IVA of first price we use the formula \texttt{A2*F\$4/100} where we fix the row of cell F4 because we wan it remain fixed when copying the formula down. Observe how the reference to cell F4 doesn't change when copying the formula down.

\textbf{Example}. This
\href{http://aprendeconalf.es/office/excel/manual/img/example_multiplication_table.gif}{animation} shows how to
calculate the multiplication table using absolute references.

In general, if you want to fix a reference in a formula that you pretend to copy horizontally, you must precede the
column name with a \$ sign; and if you pretend to copy the formula vertically, you must precede the row name with a \$
sign. 

\subsection{Naming cells and ranges}\hypertarget{naming-cells-and-ranges}{}\label{naming-cells-and-ranges}

Cell references are somewhat abstract, and don't really communicate anything about the data they contain. This makes
formulas that involve multiple references difficult to understand. To overcome this difficulty Excel allows to give name
to cells or ranges. To define a cell or range name, select or cell range and click the \texttt{Define Name} button of
the \texttt{Defined Names} panel in the ribbon's \texttt{Formulas} tab. In the dialog that appears give a
name to the cell and click OK. Cell or range names must begin with a letter and can't include spaces.

You can also set the name of a cell or range in the name box of the input bar (see figure~\ref{img-name_box}).

\begin{figure}[htbp]
\begin{center}
\includegraphics[scale=0.7]{../img/name_box.png}
\end{center}
\caption{Name box for giving names to cells or ranges.}
\label{img-name_box}
\end{figure}

After that you can use that cell o range name in any formula. Observe that references with names are always absolutes.

\textbf{Example}. This \href{http://aprendeconalf.es/office/excel/manual/img/example_formulas_with_defined_names.gif}{animation} shows how to calculate the IVA of a list of prices using a cell name for the cell that contains the IVA percentage.

\section{Functions}\label{functions}

Excel has a huge library of predefined functions that performs different calculations organised by categories. There are three ways to to enter a function in a formula expression:

\begin{itemize}
\item Type it rawly if you know its name and syntax.
\item Select it from the buttons of the \texttt{Functions Library} panel in the ribbon's \texttt{Formulas} tab.
\end{itemize}

\begin{center}
\includegraphics[scale=0.7]{../img/panel_formulas.png}
\end{center}

\begin{itemize}
\item Click the \texttt{Insert Function} button \includegraphics[scale=0.7]{../img/button_insert_function.png} from the
input bar.
This will show you a dialog where you can type some key words for looking the desired function an select it (see
figure~\ref{img-dialog_insert_function}).
This dialog also shows help about the function and its syntax.
\end{itemize}

\begin{figure}[htbp]
\begin{center}
\includegraphics[scale=0.7]{../img/dialog_insert_function.png}
\end{center}
\caption{Insert function dialog.}
\label{img-dialog_insert_function}
\end{figure}



\section{Numeric functions}\label{numericfunctions}
Numeric functions work with numbers or cells that contains numbers. They are the most frequently used. 


\subsection{SUM function}\hypertarget{sum-function}{}\label{sum-function}

The most common function is \texttt{SUM} that calculates the sum of several numbers. Its syntax is
\texttt{SUM(number1,number2,...)} where \emph{number1, number2}, etc. are the numbers or cell ranges that you want to
sum.

\textbf{Example} This \href{http://aprendeconalf.es/office/excel/manual/img/example_function_sum.gif}{animation} shows how to calculate the sum of the subject grades for every student in a course.

\subsection{SUMIF function}\hypertarget{sumif-function}{}\label{sumif-function}

The \texttt{SUMIF} function its similar to the \texttt{SUM} function but only sum numbers that satisfied a given
criterion.
Its syntax is \texttt{SUMIF(range,criterion,sum-range)} \emph{range} is the cell range to check the criterion,
\emph{criterion} is the condition expression of the criterion, \emph{sum-range} is the range with the values to sum (if
this argument is not provided, the sum is calculated over the values of the \emph{range} argument that meet the
criterion).

The expression with the condition can be a number, a cell reference, a logical expression starting with a logical
operator
(\texttt{=},\texttt{\textgreater{}},\texttt{\textless{}},\texttt{\textgreater{}=},\texttt{\textless{}=},\texttt{\textless{}\textgreater{}})
in double quotes, or a pattern text with wildcards like the question mark \texttt{?} (that matches any
character) or the asterisk \texttt{*} (that matches any character string) in double quotes. 

\textbf{Example} This \href{http://aprendeconalf.es/office/excel/manual/img/example_function_sumif.gif}{animation} shows how to calculate the sum of the grades greater than or equal to 5 for every student in a course.

\subsection{COUNT function}\hypertarget{count-function}{}\label{count-function}

The \texttt{COUNT} function counts the number of cells with numbers in a range.  Its syntax is
\texttt{COUNT(value1,value2,...)} where \emph{value1, value2}, etc. are the values or cell ranges to count.

\textbf{Example} This \href{http://aprendeconalf.es/office/excel/manual/img/example_function_count.gif}{animation} shows how to calculate the number of subjects grades for every student in a course.

\subsection{COUNTIF function}\hypertarget{countif-function}{}\label{countif-function}

The \texttt{COUNTIF} function its similar to the \texttt{COUNT} but only counts number of cells that satisfied a given criterion. Its syntax is \texttt{SUMIF(range,criterion)} \emph{range} is the cell range to check the criterion and \emph{criterion} is the condition expression of the criterion,.

The expression with the condition can be a number, a cell reference, a logical expression starting with a logical
operator
(\texttt{=},\texttt{\textgreater{}},\texttt{\textless{}},\texttt{\textgreater{}=},\texttt{\textless{}=},\texttt{\textless{}\textgreater{}})
in double quotes, or a pattern text with wildcards like the question mark \texttt{?} (that matches any character)
or the asterisk \texttt{*} (that matches any character string) in double quotes.

\textbf{Example} This \href{http://aprendeconalf.es/office/excel/manual/img/example_function_countif.gif}{animation} shows how to calculate the number of passed subjects (grade greater than or equal to 5).

\subsection{MIN function}\hypertarget{min-function}{}\label{min-function}

The \texttt{MIN} function calculates the minimum value of several numbers. Its syntax is
\texttt{MIN(number1,number2,...)} where \emph{number1, number2}, etc. are numbers or cell ranges for which you want the
minimum.

\textbf{Example} This \href{http://aprendeconalf.es/office/excel/manual/img/example_function_min.gif}{animation} shows how to calculate the minimum grade for every student in a course.

\subsection{MAX function}\hypertarget{max-function}{}\label{max-function}

The \texttt{MAX} function calculates the maximum value of several numbers. Its syntax is
\texttt{MAX(number1,number2,...)} where \emph{number1, number2}, etc. are numbers or cell ranges for which you want the
maximum.

\textbf{Example} This \href{http://aprendeconalf.es/office/excel/manual/img/example_function_max.gif}{animation} shows
how to calculate the maximum grade for every student in a course.


\subsection{ISNUMBER function}
\label{isnumberfunction}

The \texttt{ISNUMBER} function checks if a value is number or not and returns the logical value TRUE in the first case and FALSE in the second. Its syntax is \texttt{ISNUMBER(value)} where \emph{value} is a value or a cell reference.

\textbf{Example} This \href{http://aprendeconalf.es/office/excel/manual/img/example_function_isnumber.gif}{animation}
shows how to check if the cells of a range contain numbers or not. Observe that in the example cells with numbers are aligned to the right and that dates are numbers.



\section{Logical functions}\hypertarget{logical-functions}{}\label{logical-functions}

Logical functions are very useful to take decisions.

\subsection{IF function}\hypertarget{if-function}{}\label{if-function}

The most important logical function is the \texttt{IF} functions, that checks whether a condition is met and returns a
value if is true or another value if is false. Its syntax is \texttt{IF(condition,true\_value,false\_value)}, where
\emph{condition} is the logical condition to test, \emph{true\_value} is the returned value if the condition is true,
and \emph{false\_value} is the returned value if the condition is false.

In the logical condition expression you use logical operators like equal \texttt{=}, not equal
\texttt{\textless{}\textgreater{}}, greater \texttt{\textgreater{}}, less \texttt{\textless{}}, greater than or equal to
\texttt{\textgreater{}=}, less than or equal to \texttt{\textless{}=}, etc. In the true or false value you can put
numbers, text in double quotes, dates, cell references or other formulas.

\textbf{Example} This \href{http://aprendeconalf.es/office/excel/manual/img/example_function_if.gif}{animation} shows how to use the IF function to decide if students pass or don't pass a course depending on whether the average grade is greater than or equal to 5.

\subsection{AND function}\hypertarget{and-function}{}\label{and-function}

The \texttt{AND} function will return TRUE if all its arguments are true and FALSE if at least one argument is false. Its syntax is \texttt{AND(contidion1,condition2,...)}, where \emph{condition1, condition2,} etc are logical conditions.

The following table, known as a \emph{truth table}, shows the returned value by the AND function according to the
corresponding values of its arguments.

\begin{longtable}{|c|c|c|}
\hline
A & B & AND(A,B)\\
\hline
TRUE & TRUE & TRUE\\
TRUE & FALSE & FALSE\\
FALSE & TRUE & FALSE\\
FALSE & FALSE & FALSE\\
\hline
\end{longtable}

~{}

\textbf{Example}. This \href{http://aprendeconalf.es/office/excel/manual/img/example_function_and.gif}{animation} shows how to use the AND function to see which students have passed all the subjects of a course with a grade greater than or equal to 5. Observe that conditions that involve blank cells are always false.

\subsection{OR function}\hypertarget{or-function}{}\label{or-function}

The \texttt{OR} function will return TRUE if one or more of its arguments are true and FALSE if all its arguments are false. Its syntax is \texttt{OR(contidion1,condition2,...)}, where \emph{condition1, condition2,} etc are logical conditions.

The following truth table shows the returned value by the OR function according to the corresponding values of its
arguments.

\begin{longtable}{|c|c|c|}
\hline
A & B & OR(A,B)\\
\hline
TRUE & TRUE & TRUE\\
TRUE & FALSE & TRUE\\
FALSE & TRUE & TRUE\\
FALSE & FALSE & FALSE\\
\hline
\end{longtable}

~{}

\textbf{Example}. This \href{http://aprendeconalf.es/office/excel/manual/img/example_function_or.gif}{animation} shows how to use the OR function to see which students have not passed some subjects of a course with a grade greater than or equal to 5.

\subsection{NOT function}\hypertarget{not-function}{}\label{not-function}

The \texttt{NOT} function will return TRUE if its argument is FALSE, and FALSE if its argument is TRUE. Its syntax is \texttt{NOT(condition)}, where \emph{condition} is a logical condition.

The following truth table shows the returned value by the NOT function according to the corresponding values of its
argument.

\begin{longtable}{|c|c|}
\hline
A & NOT(A)\\
\hline
TRUE & FALSE\\
FALSE & TRUE\\
\hline
\end{longtable}

~{}

\section{Date and time functions}\hypertarget{date-and-time-functions}{}\label{date-and-time-functions}

Date and time functions performs operations with dates and times respectively.

Excel convert automatically any entry with with a date or time formats into a serial number. For dates, this serial number represents the number of days that have elapsed since the beginning of the twentieth century (so that January 1, 1900, is serial number 1; January 2, 1900, is serial number 2; and so on). For times, this serial number is a fraction that represents the number of hours, minutes, and seconds that have elapsed since midnight (so that  00:00:00 is serial number 0.00000000, 12:00:00 p.m. (noon) is serial number 0.50000000; 11:00:00 p.m. is 0.95833333; and so on).

\subsection{Time elapsed between two dates or times.}\hypertarget{time-elapsed-between-two-dates-or-times}{}\label{time-elapsed-between-two-dates-or-times}

To calculate the time elapsed between two dates or times, just enter a formula that subtracts the earlier date or time from the later date or time.
In the case of dates, Excel will return the number of days between these dates. If you want to express it in year units, just divide the number of days by 365.25. In the case of times, Excel will return the number of hours between these times. If you want to express it in days unit, just change the cell format to General.

\textbf{Example}. This \href{http://aprendeconalf.es/office/excel/manual/img/example_time_elapsed.gif}{animation} shows how to calculate the time elapsed between two dates and two times.

\subsection{TODAY function}\hypertarget{today-function}{}\label{today-function}

The function \texttt{TODAY} returns the system date (usually the current date). Its syntax is \texttt{TODAY()} and this functions doesn't have arguments.

\textbf{Example}. This \href{http://aprendeconalf.es/office/excel/manual/img/example_function_today.gif}{animation} shows how to calculate current age of a person using the TODAY function.

\subsection{DATE function}\hypertarget{date-function}{}\label{date-function}

The function \texttt{DATE} returns a date serial number for the date specified by the year, month, and day argument. Its syntax is \texttt{DATE(year,month,day)}, where \emph{year} is the year, \emph{month} is the month (in number) and \emph{day} is the day.

\textbf{Example}. This \href{http://aprendeconalf.es/office/excel/manual/img/example_function_date.gif}{animation} shows how to calculate the date given the year, moth and day.

\subsection{DAY, WEEKDAY, MONTH and YEAR functions}\hypertarget{day-weekday-month-and-year-functions}{}\label{day-weekday-month-and-year-functions}

The \texttt{DAY} function returns the day of the month of a date. Its' syntax is \texttt{DAY(date)}, where \emph{date} is the serial number of the date.

The \texttt{WEEKDAY} function returns the day of the week of a date. Its' syntax is \texttt{WEEKDAY(date,type)}, where \emph{date} is the serial number of the date and \emph{type} has three possible values (1: 1 equals Sunday and 7 Saturday, 2: 1 equals Monday and 7 equals Sunday; 3: 0 equals Monday and 6 equals Sunday).

The \texttt{MONTH} function returns the number of the month of a date. Its' syntax is \texttt{MONTH(date)}, where \emph{date} is the serial number of the date.

The \texttt{YEAR} function returns the year of a date. Its' syntax is \texttt{YEAR(date)}, where \emph{date} is the serial number of the date.

\textbf{Example}. This \href{http://aprendeconalf.es/office/excel/manual/img/example_function_day.gif}{animation} shows how to calculate the day, week day, month and year of a date.

\subsection{NOW function}\hypertarget{now-function}{}\label{now-function}

The function \texttt{NOW} returns the system time (usually the current time). Its syntax is \texttt{NOW()} and this functions doesn't have arguments.

\textbf{Example}. This \href{http://aprendeconalf.es/office/excel/manual/img/example_function_now.gif}{animation} shows how to calculate current age of a person using the TODAY function.

\subsection{TIME function}\hypertarget{time-function}{}\label{time-function}

The function \texttt{TIME} returns a time serial number for the time specified by the hours, minutes and seconds argument. Its syntax is \texttt{TIME(hours,minutes,seconds)}, where \emph{year} is the year, \emph{month} is the month (in number) and \emph{day} is the day.

\textbf{Example}. This \href{http://aprendeconalf.es/office/excel/manual/img/example_function_time.gif}{animation} shows how to calculate the date given the year, moth and day.

\subsection{HOUR, MINUTE and SECOND functions}\hypertarget{hour-minute-and-second-functions}{}\label{hour-minute-and-second-functions}

The \texttt{HOUR} function returns the hour of a time. Its' syntax is \texttt{HOUR(time)}, where \emph{time} is the serial number of the time.

The \texttt{MINUTE} function returns the minute of a time. Its' syntax is \texttt{MINUTE(time)}, where \emph{time} is the serial number of the time.

The \texttt{SECOND} function returns the hour of a time. Its' syntax is \texttt{SECOND(time)}, where \emph{time} is the serial number of the time.

\textbf{Example}. This \href{http://aprendeconalf.es/office/excel/manual/img/example_function_hour.gif}{animation} shows how to calculate the hour, minute and second of a time.



\section{Text functions}\label{textfunctions}
Text functions performs different actions on text data type.

\subsection{TEXT function}\label{textfunction}
The \texttt{TEXT} function converts a number into text using a format specified by the users. Its syntax is \texttt{TEXT(number,format)} where \emph{number} is a number or a cell reference that you want to convert to text, and \emph{format} is the format pattern for the text in double quotes. In that pattern you can use a \texttt{0} for numbers, \texttt{.} for decimal separator, \texttt{d} for days, \texttt{m} for months, \texttt{y} years, \texttt{h} for hours, \texttt{m} for minutes and \texttt{s} for seconds. Also you can use currency signs and the percentage sign \texttt{\%}. 

\textbf{Example} This \href{http://aprendeconalf.es/office/excel/manual/img/example_function_text.gif}{animation} shows
how to convert different numbers, dates and times to text.


\subsection{VALUE function}\label{valuefunction}
The \texttt{VALUE} function converts a text string into a number. Its syntax is \texttt{VALUE(text)} where \emph{text} is a text or a cell reference with text that represents a number.

\textbf{Example} This \href{http://aprendeconalf.es/office/excel/manual/img/example_function_value.gif}{animation} shows
how to convert different text strings representing numbers, times and percentages to numbers.


\subsection{T function}\label{tfunction}
The \texttt{T} function checks if a value is text and if so, returns the text; Otherwise, the function returns an empty text string. Its syntax is \texttt{T(value)} where \emph{value} is a value or a cell reference.

\textbf{Example} This \href{http://aprendeconalf.es/office/excel/manual/img/example_function_t.gif}{animation} shows
how to check if the cells of a range contain text or not. Observe that in the example cells with text are aligned to the left.


\subsection{ISTEXT function}\label{istextfunction}
The \texttt{ISTEXT} function checks if a value is text or not and returns the logical value TRUE in the first case and FALSE in the second. Its syntax is \texttt{ISTEXT(value)} where \emph{value} is a value or a cell reference.

\textbf{Example} This \href{http://aprendeconalf.es/office/excel/manual/img/example_function_istext.gif}{animation}
shows how to check if the cells of a range contain text or not. Observe that in the example cells with text are aligned to the left.


\subsection{LEN function}\label{lenfunction}
The \texttt{LEN} function counts the number of characters of a text string. Its syntax is \texttt{LEN(text)} where \emph{text} is a text string or a cell reference with text.

\textbf{Example} This \href{http://aprendeconalf.es/office/excel/manual/img/example_function_len.gif}{animation} shows
how to count the number of characters of several words. Observe that numbers are previously converted to text, and that blank cells have 0 characters.


\subsection{CONCATENATE function}\label{concatenatefunction}
The \texttt{CONCATENATE} function joins together two or more text strings into a combined text string. Its syntax is \texttt{CONCATENATE(text1,text2,...)} where \emph{text1, text2, {\ldots}} are text strings or cell ranges with text to join.

\textbf{Example} This \href{http://aprendeconalf.es/office/excel/manual/img/example_function_concatenate.gif}{animation}
shows how to concatenate the first name and the last name of some persons with a blank space between them.


\subsection{FIND and SEARCH functions}\label{findandsearchfunctions}
The \texttt{FIND} function returns the position of a specified character or sub-string within a given text string. Its syntax is \texttt{FIND(find\_text,within\_text,[start\_num])} where \emph{find\_text} is the sub-string to find, \emph{within\_text} is text where to find the sub-string, and \emph{start\_num} is an optional argument that specifies the position in the \emph{within\_text} string, from which the search should begin (if omitted the search starts from the first character). The search is case-sensitive. 

The \texttt{SEARCH} functions works the same that the \texttt{FIND} function except that is not case-sensitive. 

\textbf{Example} This \href{http://aprendeconalf.es/office/excel/manual/img/example_function_find_search.gif}{animation}
shows how to calculate the position of some text sub-strings in a text with the FIND and the SEARCH functions.


\subsection{SUBSTITUTE functions}\label{substitutefunctions}
The \texttt{SUBSTITUTE} function replaces one or more instances of a specified text sub-string with another one supplied within a given text string. Its syntax is \texttt{SUBSTITUTE(text, old\_text, new\_text, [instance\_num])} where \emph{text} is the text where to perform the substitution, \emph{old\_text} is the sub-string to replace, \emph{new\_text} is the new text string that it is used to replace the \emph{old\_text} string, and \emph{instance\_num} is an optional argument that specifies which occurrence of the \emph{old\_text} should be replaced by the \emph{new\_text} (if this argument is not specified all instances of \emph{old\_text} are replaced with the \emph{new\_text}). The search is case-sensitive. 

\textbf{Example} This \href{http://aprendeconalf.es/office/excel/manual/img/example_function_substitute.gif}{animation}
shows how to replace some sub-strings in some texts by other text strings.


\subsection{LOWER and UPPER functions}\label{lowerandupperfunctions}
The \texttt{LOWER} function converts all characters in a text string to lower case. Its syntax is \texttt{LOWER(text)} where \emph{text} is the text to convert to lower case.

The \texttt{UPPER} functions works like the \texttt{LOWER} function but it converts text to upper case. 

\textbf{Example} This \href{http://aprendeconalf.es/office/excel/manual/img/example_function_lower_upper.gif}{animation}
shows how to convert to lower case some text strings.



\section{Database functions}

See the section~\ref{database-functions}.

\section{Mathematical functions}\hypertarget{mathematical-functions}{}\label{mathematical-functions}

Some common mathematical functions included in the function library are exponentials, logarithmic and trigonometric.

\subsection{SQRT function}\hypertarget{sqrt-function}{}\label{sqrt-function}

The \texttt{SQRT} function calculates the root square of a number. Its syntax is \texttt{SQRT(number)} where \emph{number} is a number or a cell reference for which you want the square root.

\textbf{Example} This \href{http://aprendeconalf.es/office/excel/manual/img/example_function_sqrt.gif}{animation} shows how to calculate the square root of grades in a course.

\subsection{EXP function}\hypertarget{exp-function}{}\label{exp-function}

The \texttt{EXP} function calculates the exponential of a number. Its syntax is \texttt{EXP(number)} where \emph{number} is a number or a cell reference for which you want the exponential.

\textbf{Example} This \href{http://aprendeconalf.es/office/excel/manual/img/example_function_exp.gif}{animation} shows how to calculate the exponential of grades in a course.

\subsection{LN and LOG functions}\hypertarget{ln-and-log-functions}{}\label{ln-and-log-functions}

The \texttt{LN} function calculates the natural logarithm of a number (that is with base $e$). Its syntax is \texttt{LN(number)} where \emph{number} is a number or a cell reference for which you want the natural logarithm.

The \texttt{LOG} function calculates the logarithm of a number in a given base. Its syntax is \texttt{LOG(number,[base])} where \emph{number} is a number or a cell reference for which you want the logarithm and \emph{base} is the base of the logarithm (if this argument is omitted, then base 10 is taken).

\textbf{Example} This \href{http://aprendeconalf.es/office/excel/manual/img/example_function_ln.gif}{animation} shows how to calculate the natural logarithm and the base 10 logarithm of grades in a course.

\subsection{PI function}\hypertarget{pi-function}{}\label{pi-function}

The \texttt{PI} function returns the constant value of $\pi$. Its syntax is \texttt{PI()} without arguments.

\subsection{SIN, COS and TAN functions}\hypertarget{sin-cos-and-tan-functions}{}\label{sin-cos-and-tan-functions}

The \texttt{SIN} function calculates the sine of an angle in radians. Its syntax is \texttt{SIN(angle)} where \emph{angle} is a number or a cell reference with the radians for which you want the sine.

The \texttt{COS} function calculates the cosine of an angle in radians. Its syntax is \texttt{COS(angle)} where \emph{angle} is a number or a cell reference with the radians for which you want the cosine.

The \texttt{TAN} function calculates the tangent of an angle in radians. Its syntax is \texttt{TAN(angle)} where \emph{angle} is a number or a cell reference with the radians for which you want the tangent.

If angles are in degrees, they have to be converted to radians before with the function \texttt{RADIANS(degrees)} where \emph{degrees} is a number or a cell reference with the degrees that you want to convert to radians.

\textbf{Example} This \href{http://aprendeconalf.es/office/excel/manual/img/example_function_sin_cos_tan.gif}{animation} shows how to calculate the sine, cosine and tangent of several angles. Observe that the sine of an angle o 180 degrees is not exactly 0 because the RADIANS function does not calculate the radians corresponding to a number of degrees with total accuracy.

\subsection{ROUND function}\hypertarget{round-function}{}\label{round-function}

The \texttt{ROUND} function rounds a number to a specified number of digits. Its syntax is \texttt{ROUND(number,digits)} where \emph{number} is a number or a cell reference that you want to round and \emph{digits} is the number of digits to which you want to round the number.

\textbf{Example} This \href{http://aprendeconalf.es/office/excel/manual/img/example_function_round.gif}{animation} shows how to round the grades in a course.

\subsection{ABS function}\hypertarget{abs-function}{}\label{abs-function}

The \texttt{ABS} function calculates the absolute value of a number. Its syntax is \texttt{ABS(number)} where \emph{number} is a number or a cell reference for which you want the absolute value.

\section{Statistical functions}\hypertarget{statistical-functions}{}\label{statistical-functions}

Excel provides functions to calculate the main descriptive statistics, probability distributions and also to make inferences about the population. 
For an introductory text to Statistics visit the \href{/estadistica/manual/}{Statistic manual}.

\subsection{AVERAGE function}\hypertarget{average-function}{}\label{average-function}

The \texttt{AVERAGE} function calculates the arithmetic mean of several numbers. Its syntax is \texttt{AVERAGE(number1,number2,...)} where \emph{number1,number2}, etc. are the numbers or cell ranges for which you want the average.

\textbf{Example} This \href{http://aprendeconalf.es/office/excel/manual/img/example_function_average.gif}{animation} shows how to calculate the average grade for every student in a course. Observe that the average grade is well calculated even when there are blank cells in the range.

\subsection{AVERAGEIF function}\hypertarget{averageif-function}{}\label{averageif-function}

The \texttt{AVERAGEIF} function calculates the arithmetic mean of numbers in a cell range that meet a given criterion. Its syntax is \texttt{AVERAGEIF	(range,criterion,[average-range])} where \emph{range} is the cell range to check the criterion, \emph{criterion} is the condition expression of the criterion, \emph{average-range} is the range with the values to average (if this argument is not provided, the average is calculated over the values of the \emph{range} argument that meet the criterion).

The expression with the condition can be a number, a cell reference, a logical expression starting with a logical
operator
(\texttt{=},\texttt{\textgreater{}},\texttt{\textless{}},\texttt{\textgreater{}=},\texttt{\textless{}=},\texttt{\textless{}\textgreater{}})
in double quotes, or a pattern text with wildcards like the question mark \texttt{?} (that matches any character)
or the asterisk \texttt{*} (that matches any character string) in double quotes.

\textbf{Example} This \href{http://aprendeconalf.es/office/excel/manual/img/example_function_averageif.gif}{animation} shows how to calculate the average grade of students with a grade greater than or equal to 5 for every subject in a course.

\subsection{MEDIAN function}\hypertarget{median-function}{}\label{median-function}

The \texttt{MEDIAN} function calculates the median of several numbers. Its syntax is \texttt{MEDIAN(number1,number2,...)} where \emph{number1,number2}, etc. are the numbers or cell ranges for which you want the median.

\textbf{Example} This \href{http://aprendeconalf.es/office/excel/manual/img/example_function_median.gif}{animation} shows how to calculate the median grade for every student in a course. Observe that the median grade is well calculated even when there are blank cells in the range.

\subsection{MODE function}\hypertarget{mode-function}{}\label{mode-function}

The \texttt{MODE} function calculates the mode of several numbers. Its syntax is \texttt{MODE(number1,number2,...)} where \emph{number1,number2}, etc. are the numbers or cell ranges for which you want the mode.

\textbf{Example} This \href{http://aprendeconalf.es/office/excel/manual/img/example_function_mode.gif}{animation} shows how to calculate the mode grade for every student in a course. Observe that the mode grade is not calculated when there are not repetitions of values.

\subsection{PERCENTILE.EXC function}\hypertarget{percentileexc-function}{}\label{percentileexc-function}

The \texttt{PERCENTILE.EXC} function calculates the k-th percentile of numbers in a cell range. Its syntax is \texttt{PERCENTILE.EXC(range,k)} where \emph{range} is the cell range with the values for which you want the percentile, and \emph{k} is the relative frequency (between 0 and 1) of the percentile.

\textbf{Example} This \href{http://aprendeconalf.es/office/excel/manual/img/example_function_percentile.gif}{animation} shows how to calculate the quartiles (percentiles 25, 50 and 75) of grades for every student in a course. Observe that if we use a cell reference for the \emph{k} argument, putting a relative frequency in that cell (0.25 for first quartile, 0.5 for second quartile and 0.75 for third quartile) we get the correspondent percentile.

\subsection{VAR.P function}\hypertarget{varp-function}{}\label{varp-function}

The \texttt{VAR.P} function calculates the variance of several numbers. Its syntax is \texttt{VAR.P(number1,number2,...)} where \emph{number1,number2}, etc. are the numbers or cell ranges for which you want the variance.

\textbf{Example} This \href{http://aprendeconalf.es/office/excel/manual/img/example_function_varp.gif}{animation} shows how to calculate the variance of grades for every student in a course. Observe that the variance is well calculated even when there are blank cells in the range.

\subsection{STDEV.P function}\hypertarget{stdevp-function}{}\label{stdevp-function}

The \texttt{STDEV.P} function calculates the standard deviation of several numbers. Its syntax is \texttt{STDEV.P(number1,number2,...)} where \emph{number1,number2}, etc. are the numbers or cell ranges for which you want the standard deviation.

\textbf{Example} This \href{http://aprendeconalf.es/office/excel/manual/img/example_function_stdevp.gif}{animation} shows how to calculate the standard deviation of grades for every student in a course. Observe that you can also calculate the standard deviation applying the square root to the variance.

\subsection{SKEW function}\hypertarget{skew-function}{}\label{skew-function}

The \texttt{SKEW} function calculates the skewness coefficient of several numbers. Its syntax is \texttt{SKEW(number1,number2,...)} where \emph{number1,number2}, etc. are the numbers or cell ranges for which you want the skewness coefficient. Excel 2010 uses the following formula to calculate skewness:

\begin{displaymath}
g_1=\frac{n}{(n-1)(n-2)}\sum \left(\frac{x_i-\bar x}{s}\right)^3,
\end{displaymath}

where $\bar x$ is the mean and $s$ is the standard deviation.

\textbf{Example} This \href{http://aprendeconalf.es/office/excel/manual/img/example_function_skew.gif}{animation} shows how to calculate the skewness coefficient of grades for every subject in a course.

\subsection{KURT function}\hypertarget{kurt-function}{}\label{kurt-function}

The \texttt{KURT} function calculates the kurtosis coefficient of several numbers. Its syntax is \texttt{KURT(number1,number2,...)} where \emph{number1,number2}, etc. are the numbers or cell ranges for which you want the kurtosis coefficient. Excel 2010 uses the following formula to calculate kurtosis:

\begin{displaymath}
g_1=\frac{n(n+1)}{(n-1)(n-2)(n-3)}\sum \left(\frac{x_i-\bar x}{s}\right)^4 - \frac{3(n-1)^2}{(n-2)(n-3)},
\end{displaymath}

where $\bar x$ is the mean and $s$ is the standard deviation.

\textbf{Example} This \href{http://aprendeconalf.es/office/excel/manual/img/example_function_kurt.gif}{animation} shows how to calculate the kurtosis coefficient of grades for every subject in a course.



\section{Other functions}\label{otherfunctions}
Other common functions are the following. 


\subsection{ISBLANK function}\label{isblankfunction}
The \texttt{ISBLANK} function checks if a value is null or a cell is blank. Its syntax is \texttt{ISBLANK(value)} where \emph{value} is a value or a cell reference. 

\textbf{Example} This \href{http://aprendeconalf.es/office/excel/manual/img/example_function_isblank.gif}{animation}
shows how to check if some cells are blank or not. Observe that cell A3 is not blank because it contains a blank space.


\subsection{ISERROR function}\label{iserrorfunction}
The \texttt{ISBLANK} function checks if a value or cell is an error. Its syntax is \texttt{ISERROR(value)} where \emph{value} is a value or a cell reference. 

\textbf{Example} This \href{http://aprendeconalf.es/office/excel/manual/img/example_function_iserror.gif}{animation}
shows how to check if some cells have errors.


% <!--
% ## Financial functions
% Excel contains a bunch of financial functions for determining such things as the present and future value of an investment; the payment, number of periods, or the principal or interest part of a payment on an loan or the rate of return on an investment.
% 
% 
% # PV function
% The `PV` function returns the present value of an investment, that is the total amount that a series of future payments is worth presently. Its syntax is `PV(rate,nper,pmt,fv,style)`.   
% 
% When using financial functions, keep in mind that the fv, pv, and pmt arguments can be positive or negative, depending on whether you’re receiving the money (as in the case of an investment) or paying out the money (as in the case of a loan). Also keep in mind that you want to express the rate argument in the same units as the nper argument, so that if you make monthly payments on a loan and you express the nper as the total number of monthly payments, as in 360 (30 x 12) for a 30-year mortgage, you need to express the annual interest rate in monthly terms as well. For example, if you pay an annual interest rate of 7.5 percent on the loan, you express the rate argument as 0.075/12 so that it is monthly as well.
% 
% For more sophisticated functions you can activate the Analysis ToolPak add-in, and you will get over 30 specialised financial functions. 
% -->


\section{Auditing formulas}\hypertarget{auditing-formulas}{}\label{auditing-formulas}

When Excel can not perform an operation or when there is an error in a formula, it shows an error. Some common errors are

\begin{itemize}
\item \textbf{\#NAME? error}. Occurs when Excel does not recognize text in a formula. Usually happens when you misspell the name of a function.
\item \textbf{\#VALUE! error}. Occurs when a formula has the wrong type of argument. Usually happens when you try to performs mathematical operations with cells that does not contain numbers.
\item \textbf{\#DIV/0! error}. Occurs when a formula tries to divide a number by 0 or an empty cell.
\item \textbf{\#REF! error}. Occurs when a formula refers to a cell that is not valid. Usually happens when a formula refers to a deleted cell.
\item \textbf{\#NUM! error}. Occurs when a formula or function contains invalid numeric values. For example when trying to calculate the square root of a negative number.
\item \textbf{\#N\slash A error} Occurs when a value is not available to a function or formula.
\end{itemize}

In complex formulas it could be difficult to detect the error. Fortunately, Excel provide some tools for tracking down errors.

\subsection{Tracing formulas}\hypertarget{tracing-formulas}{}\label{tracing-formulas}

The simplest procedure to trace formulas is double click a cell with a formula. This will show the cells referenced by the formula marked in different colours.

Another possibility is to trace precedents or dependents references. If you select a cell with a formula and click the \texttt{Trace Precedents} button of the \texttt{Formula Auditing} panel on the ribbon's \texttt{Formulas} tab, Excel will show arrows to the cells that affect the value of the selected cell. And if click the \texttt{Trace Dependents} button of the \texttt{Formula Auditing} panel on the ribbon's \texttt{Formulas} tab, Excel will show arrows to the cells that are affected by selected cell. To remove the arrow simply click the \texttt{Remove Arrows} button of the \texttt{Formula Auditing} panel on the ribbon's \texttt{Formulas} tab.

\textbf{Example} This \href{http://aprendeconalf.es/office/excel/manual/img/example_formula_trace.gif}{animation} shows how to trace a formula to calculate the price of product without discount, with discount but without taxes and with discount and taxes.

\subsection{Error checking}\hypertarget{error-checking}{}\label{error-checking}

If some formula have an error, you can check where the error come from selecting the cell with the error and clicking
the \texttt{Error Checking} button \includegraphics[scale=0.7]{../img/button_error_checking.png} of the \texttt{Formula
Auditing} panel on the ribbon's \texttt{Formulas} tab. This will display a dialog with the formula expression, an explanation of the error and several options. If the error is in the selected cell you can click the option \texttt{Show Calculation Steps} to evaluate the formula (see the section \hyperlink{formula\_evaluation}{Formula evaluation}). But if the error is in a cell that affects the selected cell you can click the option \texttt{Trace Error}. This will show red arrows to cells where the error come from.

\textbf{Example} This \href{http://aprendeconalf.es/office/excel/manual/img/example_error_checking.gif}{animation} shows how to check an error in a formula to calculate the price of product without discount, with discount but without taxes and with discount and taxes.

\subsection{Formula evaluation}\hypertarget{a-nameformulaevaluationaformula-evaluation}{}\label{a-nameformulaevaluationaformula-evaluation}

In general, you can evaluate any formula, even if it has no error, selecting the cell with the formula and clicking the
\texttt{Formula Evaluation} button \includegraphics[scale=0.7]{../img/button_evaluate_formula.png} of the
\texttt{Formula Auditing} panel on the ribbon's \texttt{Formulas} tab. This will display a dialog where you can evaluate the formula step by step.

\textbf{Example} This \href{http://aprendeconalf.es/office/excel/manual/img/example_formula_evaluation.gif}{animation} shows how to check an error in a formula to calculate the price of product without discount, with discount but without taxes and with discount and taxes.

